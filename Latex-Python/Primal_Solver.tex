\documentclass [12pt] {article}
\usepackage{amsmath}
\usepackage{url}
\usepackage[super]{nth}
\pagestyle{plain}
\begin{document}

\section{Primal Simplex Method.}
The matrix A is given by
\[
\begin{bmatrix} 1.0 & 4.0 & 0.0 \\ 3.0 & -1.0 & 1.0 \\ \end{bmatrix}.
\]
The initial sets of basic and nonbasic indices are
\[
{\mathcal B} = \left \{ 4, 5 \right \}  \quad
and    \quad
{\mathcal N} = \left \{ 1, 2, 3 \right \}.
\]
Corresponding to these sets, we have the submatrices of A:
\[
B = \begin{bmatrix} 1.0 & 0.0 \\ 0.0 & 1.0 \\ \end{bmatrix}   \quad
N = \begin{bmatrix} 1.0 & 4.0 & 0.0 \\ 3.0 & -1.0 & 1.0 \\ \end{bmatrix},
\]
and the initial values of the basic variables are given by
\[
 x^*_{\mathcal B} = b = \begin{bmatrix} 1.0 \\ 3.0 \\ \end{bmatrix} ,   
\]
and the initial nonbasic dual variables are simply
\[
z^*_{\mathcal N} = -c_{\mathcal N} = \begin{bmatrix} -4.0 \\ -1.0 \\ -3.0 \\ \end{bmatrix}.
\]

\subsection{ \nth{1} Iteration.}
\textit{Step 1. } Since \textit{z}$^*_{\mathcal N}$ has some negative components, the current solution is not optimal.\\
\textit{Step 2. } Since \textit{z}$^*_1$ = -4.0 and this is the most negative of the two nonbasic dual variables, we see that the entering index is
\[
j = 1.
\]
\textit{Step 3. }
\[
\Delta X_{\mathcal B} = B^{-1} N e_j = \begin{bmatrix} 1.0 & 4.0 & 0.0 \\ 3.0 & -1.0 & 1.0 \\ \end{bmatrix}
\begin{bmatrix} 1.0 \\ 0.0 \\ 0.0 \\ \end{bmatrix}
= \begin{bmatrix} 1.0 \\ 3.0 \\ \end{bmatrix}
.
\]
\textit{Step 4. }
\[
t = \left ( max \left \{ \frac{1.0}{1.0} , \frac{3.0}{3.0} \right \} \right )^{-1}   = 1.0.
\]\textit{Step 5.} Since the ratio that achieved the maximum in Step 4 was the \nth{1} ratio and this ratio corresponds to basis index 4, we see that
\[
i = 4.
\]
\textit{Step 6. } 
\[
\Delta z_{\mathcal N} = -\left (B^{-1} N \right )^T e_i = - \begin{bmatrix} 1.0 & 3.0 \\ 4.0 & -1.0 \\ 0.0 & 1.0 \\ \end{bmatrix}
\begin{bmatrix} 1.0 \\ 0.0 \\ \end{bmatrix}
= \begin{bmatrix} -1.0 \\ -4.0 \\ -0.0 \\ \end{bmatrix}.
\]
\textit{Step 7. } 
\[
s =  \frac{z^*_j}{\Delta z_j} = \frac{ -4.0 }{ -1.0 } = 4.0.
\]
\textit{Step 8.}
\[
x^*_1 = 1.0,        x^*_{\mathcal B} = 
\begin{bmatrix} 1.0 \\ 3.0 \\ \end{bmatrix} - 1.0 \begin{bmatrix} 1.0 \\ 3.0 \\ \end{bmatrix} = \begin{bmatrix} 0.0 \\ 0.0 \\ \end{bmatrix} ,
\]
\[
z^*_4 = 4.0,        z^*_{\mathcal N} = 
\begin{bmatrix} -4.0 \\ -1.0 \\ -3.0 \\ \end{bmatrix} - 4.0 \begin{bmatrix} -1.0 \\ -4.0 \\ -0.0 \\ \end{bmatrix} = \begin{bmatrix} 0.0 \\ 15.0 \\ -3.0 \\ \end{bmatrix} .
\]
\textit{Step 9.} The new sets of basic and nonbasic indices are
\[
{\mathcal B} = \left \{ 1, 5 \right \}  \quad
and    \quad
{\mathcal N} = \left \{ 4, 2, 3 \right \}.
\]
Corresponding to these sets, we have the new basic and nonbasic submatrices of A,
\[
B = \begin{bmatrix} 1.0 & 0.0 \\ 3.0 & 1.0 \\ \end{bmatrix}  \quad
N = \begin{bmatrix} 1.0 & 4.0 & 0.0 \\ 0.0 & -1.0 & 1.0 \\ \end{bmatrix},
\]
and the new basic primal variables and nonbasic dual variables:
\[
 x^*_{\mathcal B} = 
\begin{bmatrix} x^*_1 \\ x^*_5 \end{bmatrix} = \begin{bmatrix} 1.0 \\ 0.0 \\ \end{bmatrix} ,   \quad
z^*_{\mathcal N} = 
\begin{bmatrix} z^*_4 \\ z^*_2 \\ z^*_3 \end{bmatrix} = \begin{bmatrix} 4.0 \\ 15.0 \\ -3.0 \\ \end{bmatrix}.
\]

\subsection{ \nth{2} Iteration.}
\textit{Step 1. } Since \textit{z}$^*_{\mathcal N}$ has some negative components, the current solution is not optimal.\\
\textit{Step 2. } Since \textit{z}$^*_3$ = -3.0 and this is the most negative of the two nonbasic dual variables, we see that the entering index is
\[
j = 3.
\]
\textit{Step 3. }
\[
\Delta X_{\mathcal B} = B^{-1} N e_j = \begin{bmatrix} 1.0 & 4.0 & 0.0 \\ -3.0 & -13.0 & 1.0 \\ \end{bmatrix}
\begin{bmatrix} 0.0 \\ 0.0 \\ 1.0 \\ \end{bmatrix}
= \begin{bmatrix} 0.0 \\ 1.0 \\ \end{bmatrix}
.
\]
\textit{Step 4. }
\[
t = \left ( max \left \{ \frac{0.0}{1.0} , \frac{1.0}{0.0} \right \} \right )^{-1}   = 0.0.
\]\textit{Step 5.} Since the ratio that achieved the maximum in Step 4 was the \nth{2} ratio and this ratio corresponds to basis index 5, we see that
\[
i = 5.
\]
\textit{Step 6. } 
\[
\Delta z_{\mathcal N} = -\left (B^{-1} N \right )^T e_i = - \begin{bmatrix} 1.0 & -3.0 \\ 4.0 & -13.0 \\ 0.0 & 1.0 \\ \end{bmatrix}
\begin{bmatrix} 0.0 \\ 1.0 \\ \end{bmatrix}
= \begin{bmatrix} 3.0 \\ 13.0 \\ -1.0 \\ \end{bmatrix}.
\]
\textit{Step 7. } 
\[
s =  \frac{z^*_j}{\Delta z_j} = \frac{ -3.0 }{ -1.0 } = 3.0.
\]
\textit{Step 8.}
\[
x^*_3 = 0.0,        x^*_{\mathcal B} = 
\begin{bmatrix} 1.0 \\ 0.0 \\ \end{bmatrix} - 0.0 \begin{bmatrix} 0.0 \\ 1.0 \\ \end{bmatrix} = \begin{bmatrix} 1.0 \\ 0.0 \\ \end{bmatrix} ,
\]
\[
z^*_5 = 3.0,        z^*_{\mathcal N} = 
\begin{bmatrix} 4.0 \\ 15.0 \\ -3.0 \\ \end{bmatrix} - 3.0 \begin{bmatrix} 3.0 \\ 13.0 \\ -1.0 \\ \end{bmatrix} = \begin{bmatrix} -5.0 \\ -24.0 \\ 0.0 \\ \end{bmatrix} .
\]
\textit{Step 9.} The new sets of basic and nonbasic indices are
\[
{\mathcal B} = \left \{ 1, 3 \right \}  \quad
and    \quad
{\mathcal N} = \left \{ 4, 2, 5 \right \}.
\]
Corresponding to these sets, we have the new basic and nonbasic submatrices of A,
\[
B = \begin{bmatrix} 1.0 & 0.0 \\ 3.0 & 1.0 \\ \end{bmatrix}  \quad
N = \begin{bmatrix} 1.0 & 4.0 & 0.0 \\ 0.0 & -1.0 & 1.0 \\ \end{bmatrix},
\]
and the new basic primal variables and nonbasic dual variables:
\[
 x^*_{\mathcal B} = 
\begin{bmatrix} x^*_1 \\ x^*_3 \end{bmatrix} = \begin{bmatrix} 1.0 \\ 0.0 \\ \end{bmatrix} ,   \quad
z^*_{\mathcal N} = 
\begin{bmatrix} z^*_4 \\ z^*_2 \\ z^*_5 \end{bmatrix} = \begin{bmatrix} -5.0 \\ -24.0 \\ 3.0 \\ \end{bmatrix}.
\]

\subsection{ \nth{3} Iteration.}
\textit{Step 1. } Since \textit{z}$^*_{\mathcal N}$ has some negative components, the current solution is not optimal.\\
\textit{Step 2. } Since \textit{z}$^*_2$ = -24.0 and this is the most negative of the two nonbasic dual variables, we see that the entering index is
\[
j = 2.
\]
\textit{Step 3. }
\[
\Delta X_{\mathcal B} = B^{-1} N e_j = \begin{bmatrix} 1.0 & 4.0 & 0.0 \\ -3.0 & -13.0 & 1.0 \\ \end{bmatrix}
\begin{bmatrix} 0.0 \\ 1.0 \\ 0.0 \\ \end{bmatrix}
= \begin{bmatrix} 4.0 \\ -13.0 \\ \end{bmatrix}
.
\]
\textit{Step 4. }
\[
t = \left ( max \left \{ \frac{4.0}{1.0} , \frac{-13.0}{0.0} \right \} \right )^{-1}   = 0.25.
\]\textit{Step 5.} Since the ratio that achieved the maximum in Step 4 was the \nth{1} ratio and this ratio corresponds to basis index 1, we see that
\[
i = 1.
\]
\textit{Step 6. } 
\[
\Delta z_{\mathcal N} = -\left (B^{-1} N \right )^T e_i = - \begin{bmatrix} 1.0 & -3.0 \\ 4.0 & -13.0 \\ 0.0 & 1.0 \\ \end{bmatrix}
\begin{bmatrix} 1.0 \\ 0.0 \\ \end{bmatrix}
= \begin{bmatrix} -1.0 \\ -4.0 \\ -0.0 \\ \end{bmatrix}.
\]
\textit{Step 7. } 
\[
s =  \frac{z^*_j}{\Delta z_j} = \frac{ -24.0 }{ -4.0 } = 6.0.
\]
\textit{Step 8.}
\[
x^*_2 = 0.25,        x^*_{\mathcal B} = 
\begin{bmatrix} 1.0 \\ 0.0 \\ \end{bmatrix} - 0.25 \begin{bmatrix} 4.0 \\ -13.0 \\ \end{bmatrix} = \begin{bmatrix} 0.0 \\ 3.25 \\ \end{bmatrix} ,
\]
\[
z^*_1 = 6.0,        z^*_{\mathcal N} = 
\begin{bmatrix} -5.0 \\ -24.0 \\ 3.0 \\ \end{bmatrix} - 6.0 \begin{bmatrix} -1.0 \\ -4.0 \\ -0.0 \\ \end{bmatrix} = \begin{bmatrix} 1.0 \\ 0.0 \\ 3.0 \\ \end{bmatrix} .
\]
\textit{Step 9.} The new sets of basic and nonbasic indices are
\[
{\mathcal B} = \left \{ 2, 3 \right \}  \quad
and    \quad
{\mathcal N} = \left \{ 4, 1, 5 \right \}.
\]
Corresponding to these sets, we have the new basic and nonbasic submatrices of A,
\[
B = \begin{bmatrix} 4.0 & 0.0 \\ -1.0 & 1.0 \\ \end{bmatrix}  \quad
N = \begin{bmatrix} 1.0 & 1.0 & 0.0 \\ 0.0 & 3.0 & 1.0 \\ \end{bmatrix},
\]
and the new basic primal variables and nonbasic dual variables:
\[
 x^*_{\mathcal B} = 
\begin{bmatrix} x^*_2 \\ x^*_3 \end{bmatrix} = \begin{bmatrix} 0.25 \\ 3.25 \\ \end{bmatrix} ,   \quad
z^*_{\mathcal N} = 
\begin{bmatrix} z^*_4 \\ z^*_1 \\ z^*_5 \end{bmatrix} = \begin{bmatrix} 1.0 \\ 6.0 \\ 3.0 \\ \end{bmatrix}.
\]

\subsection{ \nth{4} Iteration.}
\textit{Step 1. } Since \textit{z}$^*_{\mathcal N}$ has all nonnegative components, the current solution is optimal. The optimal objective function value is
\[
\zeta^* = 4.0x^*_1  +  1.0x^*_2  +  3.0x^*_3 = [ 10.]
\]
\end{document}